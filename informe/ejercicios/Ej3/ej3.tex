\section{Implementación de tabla de verdad con compuertas lógicas}

\begin{table}[!htb]
	\vspace{5mm} %5mm vertical space
	\begin{minipage}{.45\linewidth}
        \begin{figure}[H]
            \begin{center}
			\begin{tabular}{c|c|c|c}
                a& b & c & $f~(a,~b,~c)$ \\\hline
                0& 0 & 0 & 0 \\
                0& 0 & 1 & 1 \\
                0& 1 & 0 & 1 \\
                0& 1 & 1 & 1 \\
                1& 0 & 0 & 0 \\
                1& 0 & 1 & 1 \\
                1& 1 & 0 & 0 \\
                1& 1 & 1 & 0 
               \end{tabular}
            \end{center}
            \caption{Tabla de verdad de la función dada.}
		\end{figure}
	\end{minipage}%
	\begin{minipage}{.45\linewidth}
		\begin{figure}[H]
			\begin{center}
                \begin{Karnaughvuit}
                    \maxterms{0,2,3,7}
                    \minterms{1,4,5,6}
                    \implicant{1}{5}{green}
                    \implicantcostats{4}{6}{blue}
                    %\implicant{4}{5}{red}      //para el caso de hazards
				\end{Karnaughvuit}
				\label{table:mapaKarnaugh}
			\end{center}
        \end{figure}
        \caption{Mapa de Karnaugh.}
	\end{minipage} 
\end{table}

//hablar de los hazards y q se hace para evitarlos, mostrar mapa de karnaugh con glitches evitados

\begin{wraptable}{r}{5.5cm}
    \caption{A wrapped table going nicely inside the text.}\label{wrap-tab:1}
    \begin{tabular}{ccc}\\\toprule  
    Header-1 & Header-1 & Header-1 \\\midrule
    2 &3 & 5\\  \midrule
    2 &3 & 5\\  \midrule
    2 &3 & 5\\  \bottomrule
    \end{tabular}
    \end{wraptable} 
    {\lipsum[2] 
\par
asdfsdsfdgsdfgsfdgsfgsfg
sdfsdgslfjgajfdfbsdfsfasfg j kj  bihfdfgsf fa gsdfg sdfsdgslfjgajfdfbsdfsfasfg
Table~\ref{wrap-tab:1} is a wrapped table.




